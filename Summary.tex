\documentclass[12pt]{extarticle}
%Some packages I commonly use.
\usepackage[english]{babel}
\usepackage{graphicx}
\usepackage{framed}
\usepackage[normalem]{ulem}
\usepackage{amsmath}
\usepackage{amsthm}
\usepackage{amssymb}
\usepackage{amsfonts}
\usepackage{enumerate}
\usepackage[utf8]{inputenc}
\usepackage[top=1 in,bottom=1in, left=1 in, right=1 in]{geometry}
\usepackage{xeCJK}
\usepackage{physics}

% \graphicspath{ {./images/} }


%A bunch of definitions that make my life easier
\newcommand{\matlab}{{\sc Matlab} }
\newcommand{\cvec}[1]{{\mathbf #1}}
\newcommand{\rvec}[1]{\vec{\mathbf #1}}
\newcommand{\ihat}{\hat{\textbf{\i}}}
\newcommand{\jhat}{\hat{\textbf{\j}}}
\newcommand{\khat}{\hat{\textbf{k}}}
\newcommand{\minor}{{\rm minor}}
% \newcommand{\trace}{{\rm trace}}
\newcommand{\spn}{{\rm Span}}
\newcommand{\rem}{{\rm rem}}
\newcommand{\ran}{{\rm range}}
\newcommand{\range}{{\rm range}}
\newcommand{\mdiv}{{\rm div}}
\newcommand{\proj}{{\rm proj}}
\newcommand{\R}{\mathbb{R}}
\newcommand{\N}{\mathbb{N}}
\newcommand{\Q}{\mathbb{Q}}
\newcommand{\Z}{\mathbb{Z}}
\newcommand{\<}{\langle}
\renewcommand{\>}{\rangle}
\renewcommand{\emptyset}{\varnothing}
\newcommand{\attn}[1]{\textbf{#1}}
\theoremstyle{definition}
\newtheorem{theorem}{Theorem}
\newtheorem{corollary}{Corollary}
\newtheorem*{definition}{Definition}
\newtheorem*{example}{Example}
\newtheorem*{note}{Note}
\newtheorem{exercise}{Exercise}
\newcommand{\bproof}{\bigskip {\bf Proof. }}
\newcommand{\eproof}{\hfill\qedsymbol}
\newcommand{\Disp}{\displaystyle}
\newcommand{\qe}{\hfill\(\bigtriangledown\)}

\newcommand{\SubItem}[1]{
    {\setlength\itemindent{15pt} \item[-] #1}
}


\setlength{\columnseprule}{1 pt}


\title{Nonlinear Fiber Optics\cite{agrawal_nonlinear_2007} Summary}
\author{Maodong Gao}
\date{\today}

\begin{document}

\maketitle

\section{Introduction}

    \subsection{Historical prospective}
    
        \begin{itemize}
            \item 1960, fibres lossy $\geq$ 1000dB/km. 1970 20dB/km.
            \item 1979, loss level 0.2dB/km in 1.55um. Limited by Rayleigh Scattering.
            \item 1972, Raman and Brillouin scattering process are studied using optical fibers.
            \item Stimulated other nonlinear phenomena,including optically induced birefringence, parametric 4-wave mixing, self-phase modulation
            \item 1973, soliton like pulses are suggested. 1980 observed. 6fs pulse observed in 1987.
            \item 1990s, Erbium-doped fiber amplifier, wavelength near 1550nm.
            \item 2000s, stimulated Raman scattering(so called Raman amplification), four-wave mixing(so called Fiber-optic parametric amplifiers) are two new types of amplifiers. Do not require doped atoms and spectral region not limited.
            \item Amplifiers fueled optical solitons. New types of solitons such as dispersion-managed solitons and dissipative solitons.
            \item (Chapter 11) 1978, fibre gratings. 1996, New types of fibers, such as photonic crystal fibres, crystal fibers,  holey fibers and micro-structure fibers.
            \item (Chapter 12) New types of fibers has new dispersive (GVD, group velocity dispersion) and nonlinear (relativity small core size enhance nonlinear effects) properties.
            \item (Chapter 13) Supercontinuum generation. optical spectrum of incident light broadens by a factor of $\geq$100 over a short length of fiber.
        \end{itemize}
    
    \subsection{Fiber Characteristics}
        \begin{itemize}
            \item step-index fibers, see Fig \ref{fig1.1}.
            \item Two parameters characterize optical fiber: relative core-cladding index difference; V parameter.
            \item core-cladding index difference:
                \begin{equation}
                    \Delta = \frac{n_1 - n_c}{n_1}
                    \label{core-cladding index difference}
                \end{equation}
            \item V parameter: this parameter determines number of modes supported by this fiber at a certain wavelength $\lambda$. Fiber supports single mode of a certain $\lambda$ if $V < 2.405$.
                \begin{equation}
                    V = k_0 a \sqrt{n_1^2 - n_c^2}
                    \label{V parameter}
                \end{equation}
            where a is core radius, $k_0 = \frac{2\pi}{\lambda}$
        \end{itemize}
    
        \begin{figure}[htbp]
            \centering
            \includegraphics[width=0.4\textwidth]{images/fig1.1.PNG}
            \caption{Schematic illustration of the cross section and the refractive-index profile of a stepindex fiber.}
            \label{fig1.1}
        \end{figure}
        
        \subsubsection{Material and Fabrication}
            \begin{figure}[htbp]
                \centering
                \includegraphics[width=0.6\textwidth]{images/fig1.2.PNG}
                \caption{Schematic diagram of the MCVD process commonly used for fiber fabrication.}
                \label{fig1.2}
            \end{figure}
            
            \begin{itemize}
                \item Dopants increase refractive index: $GeO_2, P_2O_5$.
                \item Dopants decrease refractive index: $boron, fluorine$
                \item Dopants to core for amplification: $ErCl_3, Nd_2O_3$
                \item MCVD(modified chemical vapor deposition) fiber fabrication process.
            \end{itemize}
            
        \subsubsection{Fiber Losses}
            Losses are characterized by attenuation constant $\alpha$:
                \begin{equation}
                    P_T = P_0 \exp{(-\alpha L)}.
                    \label{attenuation constant}
                \end{equation}
            Another unit for $\alpha$ is dB/km, which are related by:
                \begin{equation}
                    \alpha_{dB} = -\frac{10}{L}\log{(\frac{P_T}{P_0})} = 4.343\alpha
                \end{equation}
            A high loss up to 10dB/km corresponds to attenutation constant of \\
            $\alpha \approx 2 km^{-1} = 2\times 10^{-5}cm^{-1} $,
            which is already very low compared to most other materials.
            
            
            \begin{figure}[htbp]
                \centering
                \includegraphics[width=0.6\textwidth]{images/fig1.3.PNG}
                \caption{Measured loss spectrum of a single-mode silica fiber. Dashed curve shows the contribution resulting from Rayleigh scattering.}
                \label{fig1.3}
            \end{figure}
            
            Two main fiber loss channels: Rayleigh scattering; Material absorption.
            \begin{itemize}
                \item Raleigh scattering, arises from density fluctuation frozen into fused silica.Local fluctuations in refractive index, scatter lights into all directions.
                    \begin{equation}
                        \alpha_R = \frac{C_R}{\lambda^4}
                    \end{equation}
                    Raleigh scattering strength decreases rapidly as wavelength increases. $C_R$ depends on constituents of fiber core, typically around $0.7-0.9 dB/(km \mu m^4)$. As $\alpha_R$ around $0.12-0.15 dB/km$ near $\lambda = 1.55 \mu m$.
                \item Material absorption: Silica glass electronic resonance (ultraviolet region), vibrational resonance (far-infrared region beyond $2 \mu m$). Absorbs little light from $0.5 \mu m$ to $2 \mu m$.
                \item Impurity absorption: most important impurity is OH ion. Its fundamental vibrational absorption peak is at $\approx 2.73 \mu m$. The peak in Fig.\ref{fig1.3} is the second overtone of OH ions.
            \end{itemize}
            
        \subsubsection{Chromatic Dispersion}
            \begin{itemize}
                \item Chromatic dispersion basically is reflective index over optical frequency $n(\omega)$.
                \item $n(\omega)$ relation with material resonance is well approximated by Sellmeier equation:
                    \begin{equation}
                        n^2(\omega) = 1 + \sum_{j=1}^m \frac{B_j \omega_j^2}{\omega_j^2-\omega^2},
                        \label{sellmeier equation}
                    \end{equation}
                    where $\omega_j$ are material resonances frequencies, $B_j$ is the strength of $j$th resonance. $\omega_j$ contributes more as it is nearer to frequcney of interest $\omega$.
                \item For bulk-fused silica, the fitting of Sellmeier equation to $m = 3$ has the parameters in Table\ref{bulk-fused silica Sellmeier equation fitting}. This is useful when doing COMSOL simulation. The $n(\omega)$ results are shown in Fig.\ref{fig1.4}.
                    \begin{table}[htbp]
                    \centering
                        \begin{tabular}{|c|c|c|c|}
                        \hline
                        $j$ in Eq.\ref{sellmeier equation} & 1       & 2           & 3           \\ \hline
                        $B_j$                          & 0.6961663   & 0.4079426   & 0.8974794   \\ \hline
                        $\lambda_j / nm$               & 68.4043     & 116.2414    & 9896.161    \\ \hline
                        $\omega_j =2\pi c/\lambda_j$   & 2.7537E+16  & 1.62047E+16 & 1.90342E+14 \\ \hline
                        Freq /THz                      & 43.82655155 & 25.79050648 & 0.302938137 \\ \hline
                        \end{tabular}
                    \caption{bulk-fused silica Sellmeier equation(Eq.\ref{sellmeier equation}) fitting}
                    \label{bulk-fused silica Sellmeier equation fitting}
                    \end{table}
                \item Dispersion induce short optical pulse broadening.
                \item Dispersion and non-linearity are different behavior components. In nonlinear regime, these two may behave qualitatively different.
                
                \item Group index $n_g$: Define $n_g = \frac{v_g}{c}$. From mode-propagation constant $\beta$ view, Taylor expands $\beta(\omega)$ about the center pulse freq $\omega_0$, 
                    \begin{equation}
                        \beta(\omega) = n(\omega)\frac{\omega}{c}=\beta_0 + \beta_1(\omega-\omega_0)+\frac{1}{2}\beta_2(\omega-\omega_0)^2,
                        \label{propagation constant expansion}
                    \end{equation}
                    where
                    \begin{equation}
                        \beta_m = (\dv[m]{\beta}{\omega})_{\omega=\omega_0}.
                    \end{equation}
                    And $n_g$ are related to $\beta_1$ by:
                    \begin{equation}
                        \beta_1 = \frac{1}{v_g} = \frac{n_g}{c} = \frac{1}{c}(n+\omega \dv{n}{\omega}).
                        \label{group velocity and group index}
                    \end{equation}
                
                \item Group velocity dispersion (GVD): characterized by $\dv{n_g}{\omega}$, and this is related to $\beta_2$ by:
                    \begin{equation}
                        \beta_2 = \dv{\beta_1}{\omega} = \frac{1}{c}\dv{n_g}{\omega} = \frac{1}{c}(2\dv{n}{\omega} +\dv[2]{n}{\omega}).
                        \label{beta2 defination}
                    \end{equation}
                    
                \item GVD parameter D: Defined by $\dv{\beta_1}{\lambda}$. Related to $\beta_2$ by:
                    \begin{equation}
                        D = \dv{\beta_1}{\lambda} = -\frac{2\pi c}{\lambda^2}\beta_2 = -\frac{\lambda}{c}\dv[2]{n}{\lambda}
                        \label{D2 defination}
                    \end{equation}
                     The results of Eq.\ref{beta2 defination} and Eq.\ref{D2 defination} are illustrated in Fig.\ref{fig1.5}, based on the calculation of Eq.\ref{sellmeier equation}.
                     
                \item Normal and Anomalous dispersion: divided by the sign of D.
                    \begin{table}[htbp]
                    \centering
                    \begin{tabular}{|c|c|c|c|}
                        \hline
                        Normal dispersion    & $\beta_2$\textgreater{}0 & $D_2$\textless{}0    & red fast  \\ \hline
                        Anomalous dispersion & $\beta_2$\textless{}0    & $D_2$\textgreater{}0 & blue fast \\ \hline
                    \end{tabular}
                    \end{table}
                    Anomalous dispersion regime is the regime optical fibers support solitons. Solitons are self-formed by a process of balancing dispersive and nonlinear effects.
                    \begin{figure}[htbp]
                        \centering
                        \includegraphics[width=0.6\textwidth]{images/fig1.4.PNG}
                        \caption{Variation of refractive index n and group index $n_g$ with wavelength for fused silica. $n(\omega)$ is calculated from Eq.\ref{sellmeier equation}, $n_g(\omega)$ is calculaed from Eq.\ref{group velocity and group index}.}
                        \label{fig1.4}
                    \end{figure}
                    
                    \begin{figure}[htbp]
                        \centering
                        \includegraphics[width=0.6\textwidth]{images/fig1.5.PNG}
                        \caption{Variation of β2, D, and d12 with wavelength for fused silica. Both $\beta_2$ and D vanish at the zero-dispersion wavelength occurring near 1.27 $\mu m$. The $d_{12}$ here stands for $d_{12}(\lambda, 0.8\mu m)$ in Eq.\ref{d12 definition}.}
                        \label{fig1.5}
                    \end{figure}
                    
                \item Zero dispersion wavelength $\lambda_D$, where $\beta_2$ and D are 0. Around this regime $\beta_3$ (Third Order Dispersion, TOD) in Eq.\ref{propagation constant expansion} becomes important. 
                
                \item $\lambda_D$ Difference between bulked fused silica($1.27\mu m$) and actual glass fibers(typically $1.31\mu m$):
                    \SubItem{1.} Fiber core have dopants such as $GeO_2$ or $P_2O_5$.
                    \SubItem{2.} Fiber modes are confined in two directions, which discrete k number in these two directions. This geometry effect will influence dispersion.
                    \SubItem{3.} Other fiber design parameters such as core radius a and core-cladding index difference $\Delta$. Dispersion shift fibers can move $\lambda_D$ to $1.55\mu m$, where fiber loss at minium.
                    \SubItem{4.} dispersion-flattened optical fibers by multi-clad layer.
                    
                \item walk-off parameter $d_{12}$: nonlinear interaction between two optical pulses ceases
                to occur when the faster moving pulse completely walks through the slower moving pulse.
                    \begin{equation}
                        d_{12}(\lambda_1,\lambda_2) = \beta_1(\lambda_1)-\beta(\lambda_2) = v_g^{-1}(\lambda_1)-v_g^{-1}(\lambda_2)
                        \label{d12 definition}
                    \end{equation}
                    For pulses of width $T_0$, walk-off length $L_W$ defined by:
                    \begin{equation}
                        L_W = \frac{T_0}{\abs{d_{12}}}
                        \label{LW definition}
                    \end{equation}
            \end{itemize}
        
        
        \subsubsection{Polarization-Mode Dispersion}
            
            



% \subsubsection{effective red and blue detune}


% \section{Coupling mode equation}


% \section{Soliton generation}
% \subsection{modulation instability}


% \subsection{Turing pattern}
% \subsection{Bright cavity soliton}
% \subsection{Bright soliton molecules}
% \subsection{Breathing solitons}


% \section{Induction Proofs}
% \subsection{Ordinary Induction}
% \begin{exercise} Prove, for all natural numbers $n$, that 
% \begin{equation} \sum_{k=0}^n k = 1 + 2 + 3 + \cdots + n = \dfrac{n(n+1)}{2}
% \label{eq:Pn}\end{equation}
% \end{exercise}

% \begin{proof}
% We prove this by induction on $n\in\N$.  In the base case, $n=0$, and \eqref{eq:Pn} becomes
% $$\sum_{k=0}^n k = \sum_{k=0}^0 k = 0 = \dfrac{0(1)}{2} = \dfrac{n(n+1)}{2}$$

% Now, let $n>0$ be arbitrary, and assume \eqref{eq:Pn}.
% We show $\displaystyle \sum_{k=0}^{n+1} k = \dfrac{(n+1)(n+2)}{2}$.
% To that end note 
% \begin{align*}
% 	\sum_{k=0}^{n+1} k &= \left(\sum_{k=0}^{n} k\right) + (n+1) &\mbox{(sum definition)}\\
% 	&= \frac{n(n+1)}{2} + (n+1) &\mbox{(induction hypothesis)}
% 	\\
% 	&= \frac{n(n+1)}{2} + \frac{2n+2}{2} &\mbox{(common denominator)}
% 	\\
% 	&= \frac{n^2 +n}{2} + \frac{2n+2}{2} &\mbox{(distribute)}
% 	\\
% 	&= \frac{n^2 +3n + 2}{2} &\mbox{(combine like terms)}
% 	\\
% 	&= \frac{(n+1)(n+2)}{2} & \mbox{(factor the numerator)}\\
% \end{align*}

% In all cases, \eqref{eq:Pn} is true, so $\forall n\in \N$, 
% $\Disp \sum_{k=0}^n k = \dfrac{n(n+1)}{2}$
% \end{proof}
\bibliographystyle{IEEEtran}
\bibliography{references.bib}
\end{document}
